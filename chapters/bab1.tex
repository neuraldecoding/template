\subsection{Latar Belakang dan Motivasi Penelitian}

\begin{background}
Pada bagian ini, jelaskan \textit{rationale} atau motivasi yang melatarbelakangi pemilihan topik penelitian disertasi. Bagian ini dapat menjelaskan tren penelitian atau berdasarkan referensi dari topik yang menjadi minat Anda \parencite{example2023}. Sertakan juga rujukan dari referensi yang menjadi acuan latar belakang penelitian disertasi Anda \parencite{textbook2020}.

\textbf{Isi:} 100-200 kata, minimal 1 paragraf.
\end{background}

\subsection{Ringkasan Penelitian Terkait}

\begin{literaturereview}
Bagian ini berisi \textit{brief literature review}. Jelaskan penelitian-penelitian terkait hasil telaah terhadap penelitian sebelumnya yang paling mendekati dengan topik atau latar belakang disertasi Anda \parencite{conference2022}. Jelaskan juga keunggulan dan kekurangan dari penelitian sebelumnya, serta peluang penelitian lanjutan. Sertakan sitasi dari referensi utama.

\textbf{Isi:} 100-200 kata, minimal 2-3 paragraf.
\end{literaturereview}

\subsection{Potensi Gap Penelitian atau Masalah yang Akan Diangkat}

\begin{researchgap}
Pada bagian ini, tuliskan identifikasi permasalahan atau potensi gap dari fenomena yang akan diangkat dalam disertasi berdasarkan latar belakang dan penelitian terkait. Jelaskan kesenjangan yang belum teratasi oleh penelitian sebelumnya dan bagaimana penelitian Anda akan mengatasinya.

\textbf{Isi:} 100-200 kata, minimal 1 paragraf.
\end{researchgap}

\subsection{Potensi Kebaruan dari Penelitian yang Akan Dilakukan}

\begin{novelty}
Pada bagian ini, tuliskan bagaimana Anda membuktikan bahwa penelitian yang diajukan adalah baru dan orisinal (\textit{novel dan original}). Bagian ini dapat diisi dengan hasil \textit{systematic mapping study} atau \textit{meta analysis} untuk memetakan usulan topik disertasi terhadap teori atau metode terkini, sehingga dapat dibuktikan \textit{novelty} dan \textit{originality}-nya.

\textbf{Isi:} 400-500 kata, minimal 3-5 paragraf.
\end{novelty}
