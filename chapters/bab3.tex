Pada bagian ini, tuliskan minimal \textbf{3000 kata} dan maksimal \textbf{4000 kata} dalam sub-bab atau beberapa paragraf. Tuliskan secara padat dan ringkas mengenai pemahaman atau penguasaan Anda terhadap teori atau metode yang relevan dan digunakan dalam disertasi.

Misalnya, jika dalam disertasi digunakan \textit{intelligent control system} dengan \textit{Fuzzy Logic Type II}, tuliskan bagian-bagian penting tentang \textit{Fuzzy Logic Type II} yang digunakan dalam \textit{intelligent control system}.

\subsection{Teori Dasar}

Jelaskan teori-teori dasar yang menjadi fondasi penelitian Anda\cite{Li2026Adaptive}. Pastikan penjelasan mencakup:
\begin{itemize}[leftmargin=*]
    \item Definisi dan konsep fundamental
    \item Prinsip-prinsip utama
    \item Rumusan matematis yang relevan
    \item Asumsi-asumsi yang digunakan
\end{itemize}

Contoh penulisan rumus matematika menggunakan \LaTeX. Persamaan \eqref{eq:entropy} menunjukkan rumus \textit{Shannon Entropy} untuk menghitung ketidakpastian informasi:

\begin{equation}
    H(X) = - \sum_{i=1}^{n} P(x_i) \log_2 P(x_i)
    \label{eq:entropy}
\end{equation}
\myequations{Shannon Entropy}

Dimana $H(X)$ adalah entropi, $n$ adalah jumlah kemungkinan kejadian, dan $P(x_i)$ adalah probabilitas kejadian $i$.

Contoh rumus lain adalah \textit{Euclidean Distance} pada Persamaan \eqref{eq:euclidean}:

\begin{equation}
    d(p,q) = \sqrt{\sum_{i=1}^{n} (q_i - p_i)^2}
    \label{eq:euclidean}
\end{equation}
\myequations{Euclidean Distance}


\subsection{Metodologi Penelitian}

Jelaskan metodologi penelitian yang akan digunakan, mencakup pendekatan penelitian, teknik pengumpulan data, dan analisis.

\begin{figure}[htbp]
    \centering
    % Menggunakan logo-telkom.png sebagai dummy image
    \includegraphics[width=0.5\textwidth]{logo-telkom.png}
    \caption{Contoh Diagram Alir Penelitian (Dummy Figure)}
    \label{fig:alur_penelitian}
\end{figure}

Setelah Gambar \ref{fig:alur_penelitian}, jelaskan langkah-langkah detailnya:
\begin{itemize}[leftmargin=*]
    \item Pendekatan penelitian
    \item Teknik pengumpulan data
    \item Metode analisis
    \item Validasi dan verifikasi
\end{itemize}

\subsection{Alat dan Kerangka Kerja}

Jelaskan alat, kerangka kerja, atau infrastruktur yang mendukung penelitian Anda.
