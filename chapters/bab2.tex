\subsection{State-of-the-Art Penelitian}

Bagian ini menjabarkan hasil \textit{systematic literature review} terhadap referensi utama serta yang berkaitan dengan topik penelitian disertasi \parencite{lawoyin2014}. Bagian ini ditulis minimal \textbf{4000 kata} dan maksimal \textbf{5000 kata}, dalam beberapa paragraf (fleksibel).

Pada bagian ini \textbf{diakhiri dengan tabel} \textit{state-of-the-art} yang berisi ringkasan dari hasil \textit{systematic literature review}. Sumber pustaka atau referensi primer yang relevan dengan mengutamakan hasil penelitian ilmiah atau teknologi yang terkini. Disarankan penggunaan sumber pustaka 5 tahun terakhir (untuk prosiding konferensi) dan 10 tahun terakhir (bergantung pada relevansi teori atau metode) untuk jurnal atau buku teks.

Pada bagian akhir tabel, sebutkan usulan penelitian disertasi Anda untuk melihat posisi penelitian Anda dalam \textit{state-of-the-art} penelitian.

\begin{table}[htbp]
    \caption{State-of-the-Art Penelitian}
    \label{tab:sota}
    \small
    \begin{tabularx}{\textwidth}{@{} c >{\raggedright\arraybackslash}X >{\raggedright\arraybackslash}X >{\raggedright\arraybackslash}X >{\raggedright\arraybackslash}l @{}}
    \toprule
    \textbf{No} & \textbf{Peneliti} & \textbf{Objective} & \textbf{Metode \& Hasil} & \textbf{Keterangan} \\
    \midrule
    
    1 & \textcite{lawoyin2014} & 
    Analisis sistem Holter EEG 3D secara real-time. & 
    Pengolahan sinyal EEG dengan HPF, NFFT. Klasifikasi SVM. & 
    Metode dasar \\
    
    \addlinespace
    
    2 & \textcite{example2023} & 
    Meningkatkan akurasi deteksi kantuk. & 
    Menggunakan Deep Learning (CNN) dengan dataset XYZ. Akurasi 95\%. & 
    Metode modern \\
    
    \addlinespace
    
    3 & \textcite{textbook2020} & 
    Teori dasar pengukuran sinyal biomedis. & 
    Menjelaskan prinsip kerja EEG dan EKG konvensional. & 
    Referensi teori \\
    
    \bottomrule
    \end{tabularx}
\end{table}

Berdasarkan \textit{state-of-the-art} yang dibuat, berikan penjelasan tentang potensi gap penelitian yang dapat diangkat untuk penelitian disertasi Anda. Gunakan lebih dari satu tabel \textit{state-of-the-art} jika terdapat beberapa metode atau teori yang digunakan dari hasil eksplorasi Anda.
